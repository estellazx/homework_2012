\documentclass[12pt]{article}
\usepackage{latexsym}
\title{Problem Set 4}
\author{Shun Zhang (sz4554)}

\begin{document}

\maketitle

\textbf{5.1.3a}

As a set,

\begin{tabular}{l}
	bore \\
	\hline \hline
	15 \\
	16 \\
	14 \\
	18 \\
\end{tabular}

As a bag,

\begin{tabular}{l}
	bore \\
	\hline \hline
	15 \\
	16 \\
	14 \\
	16 \\
	15 \\
	15 \\
	14 \\
	18 \\
\end{tabular}

\textbf{5.1.4}

Why hold for bags:

a) The union operation on bags is just putting the rows of two bags together, without any elimination. So $(R\cup S)\cup T$ and $R\cup (S\cup T)$ are all simply putting the rows of R, S, T together.

c) $(R\Join S)\Join T$ is joining R and S first by finding $R.a=S.a$, $a$ here can be any column. Then, using this result to find $(R\Join S).b=T.b$. So the result is joining by $R.a=S.a=T.a$. Same for $R\Join (S\Join T)$.

d) The result is combining the rows in each relation, so their order doesn't matter.

Why hold for sets:

As all of these hold for bags, the operation on set does no more than eliminating same rows at each step. So these operations also hold for sets, if we eliminate same rows of the results.

\textbf{5.2.1}

a)

\begin{tabular}{l|l|l}
	A + B & $A^2$ & $B^2$ \\
	\hline \hline
	1 & 0 & 1 \\
	5 & 4 & 9 \\
	1 & 0 & 1 \\
	6 & 4 & 16 \\
	7 & 9 & 16 \\
\end{tabular}

b)

\begin{tabular}{l|l}
	B + 1 & C - 1 \\
	\hline \hline
	1 & 0 \\
	3 & 3 \\
	3 & 4 \\
	4 & 3 \\
	1 & 1 \\
	4 & 3 \\
\end{tabular}

d)

\begin{tabular}{l|l}
	B & C \\
	\hline \hline
	0 & 1 \\
	0 & 2 \\
	2 & 4 \\
	2 & 5 \\
	3 & 4 \\
	3 & 4 \\
\end{tabular}

\textbf{6.3.2}

a)

\begin{verbatim}
SELECT country
FROM Classes
WHERE numGuns =
  (SELECT max(numGuns)
   FROM Classes
  );
\end{verbatim}

\begin{verbatim}
SELECT country
FROM Classes
WHERE numGuns >= ALL
  (SELECT numGuns
   FROM Classes
  );
\end{verbatim}

b)

\begin{verbatim}
SELECT class
FROM Ships
WHERE ship IN ANY
  (SELECT ship
   FROM Outcomes
   WHERE result = 'sunk'
  );
\end{verbatim}

\begin{verbatim}
SELECT class
FROM Ships
WHERE EXISTS
  (SELECT ship
   FROM Outcomes
   WHERE name = ship AND result = 'sunk'
  );
\end{verbatim}

d)

\begin{verbatim}
SELECT name
FROM Battles
WHERE name IN
  (SELECT battle
   FROM Outcomes
   WHERE ship IN
     (SELECT name
     FROM Ships
     WHERE class = 'Kongo'
     )
  );
\end{verbatim}

\begin{verbatim}
SELECT name
FROM Battles
WHERE EXISTS
  (SELECT name
   FROM Ships, Outcomes
   WHERE name = ship AND class = 'Kongo'
  );
\end{verbatim}

\textbf{6.3.5}

a)

\begin{verbatim}
SELECT name, address
FROM MovieStar
WHERE gender = 'F' AND (name address) IN
  (SELECT name, address
   FROM MovieExec
   WHERE netWorth > 10000000
  );
\end{verbatim}

\end{document}
